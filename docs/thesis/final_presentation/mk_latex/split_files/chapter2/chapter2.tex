\chapter{基本的事項}\label{ux57faux672cux7684ux4e8bux9805}

\section{Emacs}\label{emacs}
Emacsとはプログラミング可能なテキストエディタである.
また,西谷研究室で使用を勧めているエディタである.
本研究では西谷研究室の早期学習を目的としているので,本研究で用いるエディタはEmacsとする.

\section{Ruby}\label{ruby}
Rubyとはプログラミング言語の一種である.
また,西谷研究室で使用を進めている言語である.
本研究では西谷研究室の早期学習を目的としているので,本研究で用いる言語はRubyとする.

\section{RubyGems}\label{rubygems}
RubyGemsとはRubyで用いることのできるライブラリを公開することができるサービスである.
公開されたサービスはgemコマンドを用いることで取得できる.
本研究では,RubyGemsのライブラリとして教育ツールを作成することで一般に公開する.

\section{CUI(Character User Interface)}\label{cui}
CUIとはマウス等で行う操作(GUI)を文字のみで行う方法である.
西谷研究室ではMacのTerminalで研究開発を行う.
TerminalではCUIに慣れることで作業効率が上がると考えられるので,本研究ではCUIでの操作を基本とする.

\section{使用したgemファイル}\label{ux4f7fux7528ux3057ux305fgemux30d5ux30a1ux30a4ux30eb}
\subsection{Thor}\label{thor}
Thorとはコマンドラインツール作成支援のライブラリである.
本研究で作成する教育ツールに用いる.

\subsection{RSpec}\label{rspec}
RSpecとはRuby用のテストフレームワークである.
本研究でのコード評価に用いる.

\subsection{Rubocop}\label{rubocop}
Rubocopとはコードがコーディング規約に従っているかを判定するライブラリである.
本研究でのコード評価に用いる.
