\chapter{基本的事項}\label{ux57faux672cux7684ux4e8bux9805}

\section{Emacs}\label{emacs}
本研究において使用しているeditorはEmacsである.Emacsは西谷研究室で使用を勧めているeditorであり,非常に強力なeditorとなっている.

\section{Ruby}\label{ruby}
本研究の開発言語としてRubyを使用している.理由は強力な標準ライブラリなどを持っており,構文も自由度が高く記述量も少ない.よって,縛りが少ないのでとっかかりやすくプログラミング言語の中で習熟しやすいと考えたからである.Rubyの基本的な説明は以下の通り,

\section{RubyGems}\label{rubygems}
本研究のソフトはRubyのgemと呼ばれるパッケージ管理ツールにライブラリとしてリリースしている.さらにgemを利用してファイルの操作やパスの受け渡しなどを行うパッケージを導入している.RubyGemsについての説明は以下の通り,

\section{CUI(Character User Interface)}\label{cui}
本研究はGUIではなくCUIベースで開発されている.理由はKeybindと同じで作業の高速,効率化を図るためである.CUIについての基本的な説明は以下の通り,

\section{使用したgemファイル}\label{ux4f7fux7528ux3057ux305fgemux30d5ux30a1ux30a4ux30eb}

\subsection{Thor}\label{thor}
コマンドライン作成ツールであるThorによりサブコマンドを自然言語に近い形で設定することができる.Thorについての基本的な説明は以下の通り,

\subsection{RSpec}\label{rspec}
本研究でRuby言語のソースコードとして利用した教科書ではMinitestが使われている.assert\_equalにより出力結果が正しいか判定してくれる.

\subsection{Rubocop}\label{rubocop}
自分の打ち込んだコードがコーディング規則に従っているかをチェックするのにRubocopと呼ばれるgemを使用した.Rubocopは,
